\documentclass[11pt,french]{article}
\usepackage[utf8]{inputenc}
\usepackage{titlesec}
\usepackage[hidelinks]{hyperref}
\usepackage{babel}
\usepackage[T1]{fontenc}

% Sauter une page à chaque fin (ou début?) de section
\titleclass{\section}{top}
\newcommand\sectionbreak{\clearpage}

\title{Comment découvrir son corps ?}
\author{Lucas Schwab}
\date{Mars - Aout 2020}

\begin{document}

\maketitle

% Le rapport décrit le travail effectué pendant le stage, tout en le plaçant dans son contexte.  Typiquement on s’attend à un rapport d'une taille entre 30 et 40 pages en utilisant une police de caractères de 11 points, sans compter les éventuelles annexes et les pages avant l’introduction.

% L'introduction doit présenter brièvement  

%     le sujet de stage tel que formulé initialement,
%      les modifications opérées pendant le stage,
%      les résultats obtenus pendant le stage.

% Cette partie doit être assez courte, le tout sera expliqué plus en détail dans les chapitres suivants qui doivent décrire 

%     le cadre de travail, le descriptif de l’entreprise/le laboratoire
%     le sujet et son contexte 
%     le travail réalisé (résultats obtenus, démarche et méthode suivies, difficultés rencontrées, planning, ...)
%     les conclusions.

% Il faut également fournir la liste des références bibliographiques consultées pendant le stage.  Les références doivent être aussi complètes que possible et citées dans le corps du rapport.

% Un soin particulier devra être accordé à l'orthographe, la grammaire, et la typographie.

\tableofcontents

%%%%%%%%%%%%%%%%%%%
% % % SECTION % % %
%%%%%%%%%%%%%%%%%%%
\section{Remerciements}

% \underline{Exemple d'une ancienne stagiaire}\\[10pt]
% Je tiens à remercier toute personne ayant participé de près ou de loin à ce stage.\\[10pt]
% Tout d’abord mon tuteur,  Monsieur  Boumaza Amine pour m’avoir aiguillé,  accompagné  et  pour avoir répondu à mes questions.\\[10pt]
% Le LORIA et Monsieur Marion Jean-Yves pour avoir accepté mon stage au sein du laboratoire.\\[10pt]
% L’équipe LARSEN et Monsieur Charpillet François pour avoir accepté mon stage au sein de son équipe.\\[10pt]
% Mon enseignant référent, Madame Souquières Jeanine pour avoir été là si j’avais un problème.\\[10pt]
% L’Institut  des  Sciences  du  Digital  Management  et  Cognition, Monsieur  Thomann  Laurentet les personnes ayant contribué à la partie administrativede ce stage.\\[10pt]
% Toutes les personnes m’ayant aidé lors de la recherche de ce stage et pendant toute la durée de ce dernier.


%%%%%%%%%%%%%%%%%%%
% % % SECTION % % %
%%%%%%%%%%%%%%%%%%%
\section{Introduction}
% le sujet de stage tel que formulé initialement
\subsection{Sujet}

\noindent\textbf{Motivations}

Apprendre à contrôler un robot redondant, c’est-à-dire un robot où plusieurs configurations permettent d’atteindre une position donnée, reste une tâche difficile en Intelligence Artificielle. Elle est d’autant plus difficile quand on ne dispose pas de modèle du robot. Le but de ce projet est d’explorer différentes approches pour apprendre de manière autonome ce type de contrôle.\\[10pt]
\noindent\textbf{Sujet \& Cadre du travail}

Ce projet rentre dans le cadre général de l’apprentissage artificiel et plus précisément dans le cadre où l’agent apprenant ne dispose par d’un modèle (cinématique et dynamique) de son corps. Il doit “apprendre” ce modèle en observant les effets de ses différentes actions de manière progressive, à l’image des nouveaux nés, dont les premiers gestes sont assez imprécis et s’affinent au fur et à mesure du développement de l’enfant et la découverte de leur corps.

Plusieurs algorithmes d’apprentissage proposent d’imiter le développement chez les enfant en s’appuyantsur un processus exploratoire des espaces sensoriels et moteur du robot. L’objectif du travail de recherche proposé est de proposer et d’implanter un algorithme d’exploration des effets des action moteur surun robot réel.\\[10pt]
Le protocole expérimental envisagé s’appuiera sur un robot ErgoJr à plusieurs degrés de liberté. Une caméra sera installée en bout de bras. Les algorithmes d’apprentissage devront permettre de mettre en œuvre des tâches d’asservissement visuel (parexemple, suivre “du regard” un stimulus visuel).


% les modifications opérées pendant le stage
\subsection{Travail apporté}

Nous proposons d’articuler ce projet en plusieurs étapes.
\begin{itemize}
    \item Étude bibliographique sur les méthodes de la littérature permettant le type d’apprentissage visé dans ceprojet. Cette étude pourrait être amorcée par la thèse.
    \item Mise au point d’un algorithme et expérimentations pour tests et validations.
\end{itemize}

% les résultats obtenus pendant le stage
\subsection{Résultats obtenus}

Afficher une démonstration sur robot réel

%%%%%%%%%%%%%%%%%%%
% % % SECTION % % %
%%%%%%%%%%%%%%%%%%%
% le cadre de travail, le descriptif de l’entreprise/le laboratoire
\section{Cadre de travail}

% \underline{Exemple d'une ancienne stagiaire}\\[10pt]
% Depuis 1976, le laboratoire du CRIN (Centre de Recherche en Informatique de Nancy) regroupe des chercheurs dont les problématiques se trouvent autour de deux axes: Théorie et Techniques du Logiciel (TTL) et Reconnaissance des Formes et Intelligence Artificielle (RFIA).

% Le rapprochement de l’INRIA, du CRIN et du CNRS a permis le développement de trois autres secteurs pluridisciplinaires: l’informatique et les sciences humaines et sociales, la modélisation et le calcul à hautes performances et les sciences de la vie et de la santé.

% C’est suite à tout cela que le Laboratoire lorrain de Recherche en Informatique et ses Applications (LORIA) a été créé en 1997. Il s’agit d’une Unité Mixte de Recherche (UMR) commune au CNRS, à l’Université de Lorraine et à l’INRIA. Il a pour directeur Jean-Yves Marion et Yannick Toussaint comme directeur adjoint.

% Il est composé de 30 équipes (dont 15 communes avec l’Inria), elles-mêmes structurées en 5 départements (Figure 4).

% Le 5ème département («Systèmes complexes, intelligence artificielle et robotique») est composé de 5 équipes: BISCUIT, KIWI, CAPSID, LARSEN et NEURORYTHMS. Son objectif est l’étude des systèmes complexes et leurs interactions, de l’intelligence artificielle et de la robotique.

% Ce stage se déroule au sein de l’équipe LARSEN (anciennement MaIA) qui a été créée au premier janvier 2015 et qui a pour responsable François Charpillet. Cette équipe a pour objectif de faire évoluer des robots en dehors des laboratoires ou des chaines de production.Il faut donc qu’ils puissent interagir (avec des êtres humains et d’autres robots) et être autonomes (s’adapter à des changements de leur environnement ou de leur morphologie).

%%%%%%%%%%%%%%%%%%%
% % % SECTION % % %
%%%%%%%%%%%%%%%%%%%
% le sujet et son contexte 
\section{Présentation du stage}

%%%%%%%%%%%%%%%%%%%
% % % SECTION % % %
%%%%%%%%%%%%%%%%%%%
% le travail réalisé (résultats obtenus, démarche et méthode suivies, difficultés rencontrées, planning, ...)
\section{Travail réalisé}

\subsection{Pourquoi / Modèle Géométrique Inverse}
Commencer par expliquer qu'est-ce que la cinématique directe d'un robot, puis la cinématique inverse. Expliquer la modélisation d'un robot, la création d'une modélisation géométrique directe qui peut parfois être difficile (environnement qui change, parties flexible, précision de la physique) -> utiliser le monde réel comme modélisation est utile.
Montrer les difficultés à créer une modélisation géométrique inverse. Expliquer le principe de l'exploration des espaces moteurs et sensoriels avec la comparaison à un nouveau né. L'avantage est l'autonomie de cet apprentissage ainsi que ce n'est pas limité sur le type de robot.

\subsection{Nearest Neighbor}
Montrer la construction de la base de donnée des commandes \& observations (posture / position) qui est la base de l'exploration. Le modèle géometrique inverse généré utilisera cette base avec un Nearest Neighbor. Explication du Nearest Neighbor et des manières d'enregistrer les points pour une execution plus rapide de l'algorithme.

\subsection{motor babling}
La première manière d'explorer est le motor babling. Expliquer le fonctionnement, algo simple quelque soit le robot. Donner l'exemple d'un robot simple, puis d'un robot compliqué qui n'explore pas entièrement l'espace.

\subsection{goal babling}
Pour mieux explorer l'espace on crée le goal babling, qui permet de diriger l'apprentissage. Problématique: Nearest Neighbor doit être capable d'ajouter des points dans la base en cours d'execution: exclusion de scikit sklearn.neighbors. Commencer avec un agnostic goal generation, montrer les limites de cet algorithme car explore mal l'espace si mal configuré (concentration de but sur les bords de l'espace de "travail"?)

\subsubsection{goal on grid}
Une des manières de diriger sans connaitre le modèle du robot mais pour autant sans trop déteriorer l'apprentissage, on peut partitionner l'espace et générer des buts dans une grille. Introduction de la probabilité d'exploration. But dans une zone atteinte: facile. But dans une zone non atteinte: plus difficile.

\subsubsection{frontier}
Une manière de selectionner une cellule dans une zone non atteinte.



%%%%%%%%%%%%%%%%%%%
% % % SECTION % % %
%%%%%%%%%%%%%%%%%%%
% les conclusions
\section{Conslusion}


\end{document}