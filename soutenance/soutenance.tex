\documentclass[french]{beamer}

%Coloration des frames avec mathrice.sty
\usepackage{mathrice}
\usepackage{tikz}
\usepackage{caption}
\usepackage{graphicx}
\graphicspath{{./graphics/}}

% Enlève les boutons de navigation des frames
\beamertemplatenavigationsymbolsempty

% Ajoute le logo FST sur chaque slides
\setbeamertemplate{background}{
    \begin{tikzpicture}[remember picture,overlay]
    \node[anchor=south west] at (current page.south west) {\includegraphics[height=1cm]{Logo_FST.jpg}};
\end{tikzpicture}}

% Numérotation des slides
\setbeamertemplate{footline}[frame number]

% % Frame section dans un rectangle bleu foncé
% \AtBeginSection[]{
%   \begin{frame}
%     \vfill
%     \centering
%     \begin{beamercolorbox}[sep=8pt,center,shadow=true,rounded=true]{title}
%         \usebeamerfont{title}\insertsectionhead\par
%     \end{beamercolorbox}
%     \vfill
%   \end{frame}
% }

% Page de garde
\title{Comment découvrir son corps?}
\author{Lucas SCHWAB}
\date{Mars - Août 2019}

%%%%%%%%%%%%%%%%%%%%%%%%%%%%%%%%%%%%
%%%%%%%%%%%%% Document %%%%%%%%%%%%%
%%%%%%%%%%%%%%%%%%%%%%%%%%%%%%%%%%%%
\begin{document}

% Page de garde
\begin{frame}
    \begin{center}
        \begin{beamercolorbox}[sep=8pt,center]{title}
            \usebeamerfont{title}
            \Huge \textbf{Stage Master 2}

            \huge Comment découvrir son corps?
        \end{beamercolorbox}
        \vfill
        
        Lucas SCHWAB

        \vfill

        Encadrants:

        Amine BOUMAZA
        \&
        Alain DUTECH
    \end{center}
\end{frame}

%-----------------------------------

% Sommaire
\begin{frame}
    \tableofcontents
\end{frame}

%###################################
\section{Introduction}

% Le Robot
\begin{frame}
    \frametitle{Poppy Ergo Jr}
    Présenter le robot
    Base
    Suite de moteur et de sections rigides
    Effecteur
\end{frame}

%-----------------------------------

% Modèle directe et inverse

%-----------------------------------

% Problèmes des modèles

%-----------------------------------

% Inspiration du vivant


%###################################
%section apprentissage

% Catalogue, expérience, contenu

%-----------------------------------

% Interpolation, pourquoi

%-----------------------------------

% Motor Babling

%-----------------------------------

% Goal Babling, corrige l'exploration non uniforme du motor babling
% Dirige l'apprentissage

%-----------------------------------

% Comment générer une observation à partir d'un boutons

%-----------------------------------

% Generation de but agnostique

%-----------------------------------

% Goals on grid, corrige l'approximation trop grossière de l'agnostique

%-----------------------------------



\end{document}